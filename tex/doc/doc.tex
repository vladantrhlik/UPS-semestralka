\documentclass[11pt,a4paper]{article}

\usepackage{titling}
\usepackage{float}
\usepackage[parfill]{parskip}
\usepackage[utf8]{inputenc}
\usepackage[czech]{babel}
\usepackage[T1]{fontenc}

\usepackage{amsmath}
\usepackage{amsfonts}
\usepackage{amssymb}
\usepackage{graphicx}
\usepackage[left=2cm,right=2cm,top=2cm,bottom=2cm]{geometry}
% automata
\usepackage{tikz}
\usetikzlibrary {arrows.meta,automata,positioning}

\author{Vladan Trhlík}
\title{%
	Semestrální práce z KIV/UPS \\
	\large Dots and Boxes  \\
}

\begin{document}

\maketitle

\section{Popis hry}
Hra Dots and Boxes je strategická tahová hra pro dva hráče. Herní pole tvoří čtvercová mřížka teček (v tomto případě 4 × 4), a cílem hry je propojit sousední tečky tak, aby vznikl čtverec. Hráč, který uzavře čtverec, získá bod a pokračuje dalším tahem. Hra končí, jakmile jsou všechny čtverce uzavřeny, a vítězem se stává hráč, který získal více bodů.
\section{Protokol}

\subsection{Základní popis}
Komunikace mezi klientem a serverem je zajištěna pomocí TCP protokolu. Při posílání zpráv mezi klienty a serverem se jednotlivé části zpráv oddělují znakem '\texttt{|}' a ukončovací znak je '\texttt{\textbackslash n}'. Jména hráčů a názvy her jsou omezena na velká a malá písmena a znak '\texttt{\_}'. Pokud není zpráva ve správném formátu, nebo zadané parametry nedávají v aktuálním kontextu smysl, odpovědí je zpráva \texttt{ERR<n>}, kde \texttt{n} je číslo jedné z chybových zpráv:
\begin{itemize}
	\item \texttt{1} -- invalid
	\item \texttt{2} -- name already exists
	\item \texttt{3} -- player not on turn
	\item \texttt{4} -- player not in game
	\item \texttt{5} -- already in game
	\item \texttt{6} -- max limit exceeded
\end{itemize}

\subsection{Zprávy}
\subsubsection*{Přihlášení}
\begin{itemize}
	\item \textbf{Client:} \texttt{LOGIN|<name>}
	\item \textbf{Server:} \texttt{OK / ERR<1/2/6>}
	\item \texttt{<name>}: přihlašovací jméno hráče
\end{itemize}

\subsubsection*{Načtení všech her}
\begin{itemize}
	\item \textbf{Client:} \texttt{LOAD}
	\item \textbf{Server:} \texttt{OK|<game1-name>|<game2-name>|...|<gameN-name>}
	\item \texttt{<game-name>}: název I-té hry
\end{itemize}

\subsubsection*{Vytvoření hry}
\begin{itemize}
	\item \textbf{Client:} \texttt{CREATE|<game-name>}
	\item \textbf{Server:} \texttt{OK / ERR<1/2/6>}
	\item \texttt{<game-name>}: název hry
\end{itemize}

\subsubsection*{Připojení do hry}
\begin{itemize}
	\item \textbf{Client:} \texttt{JOIN|<game-name>}
	\item \textbf{Server:} \texttt{OK|<op-name> / ERR<1/5/6>}
	\item \texttt{<game-name>}: název hry, \texttt{<op-name>}: jméno oponenta
\end{itemize}

\subsubsection*{Oponent se připojil}
\begin{itemize}
	\item \textbf{Server:} \texttt{OP\_JOIN|<op-name>}
	\item \textbf{Client:} \texttt{OK / ERR<4/5>}
	\item \texttt{<op-name>}: jméno oponenta
\end{itemize}

\subsubsection*{Oponent se odpojil}
\begin{itemize}
	\item \textbf{Server:} \texttt{OP\_LEAVE}
	\item \textbf{Client:} \texttt{OK / ERR<4>}
\end{itemize}

\subsubsection*{Odpojení od hry}
\begin{itemize}
	\item \textbf{Client:} \texttt{LEAVE}
	\item \textbf{Server:} \texttt{OK / ERR<4>}
\end{itemize}

\subsubsection*{Tah}
\begin{itemize}
	\item \textbf{Server, Client:} \texttt{TURN|<X>|<Y>}
	\item \textbf{Client, Server:} \texttt{OK / ERR<1/3>}
	\item \texttt{<X>}, \texttt{<Y>}: pozice tahu
	\item Pokud přijde zpráva ze severu, jedná se o tah oponenta, pokud od klienta, posílají se data jeho oponentovi.
\end{itemize}

\subsubsection*{Obsazení čtverce}
\begin{itemize}
	\item \textbf{Server:} \texttt{(OP\_)ACQ|<X>|<Y>[|<X2>|<Y2>]}
	\item \textbf{Client:} \texttt{OK}
	\item \texttt{<X>}, \texttt{<Y>}, \texttt{<X2>}, \texttt{<Y2>}: pozice čtverce
	\item \texttt{OP\_ACQ} informuje o obsazení oponentem, \texttt{ACQ} hráčem, který zprávu přijímá.
	\item Při propojení dvou teček může dojít k obsazení jednoho nebo dvou čtverců -- podle toho se pošle počet pozic čtverců.
\end{itemize}

\subsubsection*{Hráč je na tahu}
\begin{itemize}
	\item \textbf{Server:} \texttt{ON\_TURN}
	\item \textbf{Client:} \texttt{OK}
\end{itemize}

\subsubsection*{Oponent je na tahu}
\begin{itemize}
	\item \textbf{Server:} \texttt{OP\_TURN}
	\item \textbf{Client:} \texttt{OK}
\end{itemize}

\subsubsection*{Konec hry}
\begin{itemize}
	\item \textbf{Server:} \texttt{END|<WIN/LOSE>}
	\item \textbf{Client:} \texttt{OK}
\end{itemize}

\subsubsection*{Ping}
\begin{itemize}
	\item \textbf{Client:} \texttt{PING}, \texttt{PONG}
	\item \textbf{Server:} \texttt{PONG}, \texttt{PING}
\end{itemize}

\subsubsection*{Synchronizace stavu hry}
\begin{itemize}
	\item \textbf{Client:} \texttt{SYNC}
	\item \textbf{Server:} \texttt{OK|<w>|<h>|<stick-data>|<square-data> / ERR<4>}
	\item \texttt{<w>}, \texttt{<h>}: velikost hracího pole, \texttt{<stick-data>}: data o propojených tečkách, \texttt{<square-data>}: data o obsazených čtvercích
	\item data o propojených tečkách a obsazených čtvercích jsou řetězce z čísel 0,1,2
	\begin{itemize}
		\item 0 = nepropojeno / neobsazeno
		\item 1 = hráč který žádá o \texttt{SYNC}
		\item 2 = oponent
	\end{itemize}
\end{itemize}

\subsubsection*{Připojení po výpadku spojení}
\begin{itemize}
	\item \textbf{Client:} \texttt{RECONNECT|<name>|<game-name>}
	\item \textbf{Server:} \texttt{OK|<scene-code>}
	\item \texttt{<name>}: jméno hráče (nepovinné), \texttt{<game-name>}: název hry (nepovinné)
	\item \texttt{<scene-code>}: kód akce pro klienta (0 = login, 1 = lobby, 2 = game)
	\item Server se na základě údajů \texttt{<name>} a \texttt{<game-name>} rozhodne, zda-li se hráč připojí zpět do hry (2), nebo zůstane v lobby (1). Při neznámém jménu hráče je přesunut na nový login (0). Pokud byl hráč ve hře a připojí se zpět po více než 30 s, je automaticky přesunt do lobby.
\end{itemize}

\section{Diagram}
\begin{figure}[H]
\begin{tikzpicture}[node distance=4cm,on grid,auto] 

   \node[state,initial,initial text=$P_1$\ login] (p_0)   {lobby}; 
   \node[state,initial,initial text=$P_2$\ login] (p_1) [below=of p_0]  {lobby}; 
   \node[state] (q_0) [right=of p_0]  {waiting}; 
   \node[state] (q_1) [right=of q_0] {$P_1$ on turn}; 
   \node[state] (q_2) [right=of q_1] {$P_2$ on turn}; 
   \node[state] (q_3) [below=of q_1] {game end}; 
   \node[state] (q_5) [below=of q_3] {lobby}; 
    \path[->] 
		(p_0) edge node {create game \texttt{A}} (q_0)
		(p_1) edge [bend right] node {join game \texttt{A}} (q_1)
		(q_0) edge node {$P_2$ joined} (q_1)
		(q_1) edge [bend left] node {$P_1$ turn} (q_2)
			  edge [loop above] node {$P_1$ turn} ()
			  edge node {$P_1$ turn} (q_3)
		(q_2) edge [bend left] node {$P_2$ turn} (q_1)
			  edge [loop above] node {$P_2$ turn} ()
			  edge [bend left] node {$P_2$ turn} (q_3)
		(q_3) edge node {leave} (q_5);
	\path[->,dashed]
		(q_5) edge [bend left] node {} (p_0)
		(q_5) edge [bend left] node {} (p_1);

\end{tikzpicture}
\caption{Stavový automat představující cyklus hráče.}
\label{pic:automat}
\end{figure}

\section{Popis implementace}

\subsection{Server}
Server je napsán jazyce C za použití základních knihoven. Je rozdělen do několika modulů:
\begin{itemize}
	\item \texttt{main} -- vstupní bod programu
	\item \texttt{server} -- funkce pro spuštění serveru a obsluhy klientů
	\item \texttt{handlers} -- obslužné funkce pro jednotlivé typy zpráv
	\item \texttt{game} -- funkce pro vytváření hry, ověřování správnosti tahů apod.
	\item \texttt{utils} -- užitečné funkce pro odesílání zpráv a ověřování přechodů mezi stavy klientů
\end{itemize}
Všechny struktury jsou definovány ve \texttt{structs.h} -- obsahují strukturu \texttt{Game}, \texttt{Server} a \texttt{Player}, které slouží k uložení dat o jednlivých hráčích a hrách. Kromě toho se zde nachází i typy zpráv, výčty možných stavů a událostí, které jsou využity v přechodovém automatu (v \texttt{utils.c}).

\subsubsection{\texttt{main.c}}
Jako vstupní bod programu slouží \texttt{main.c}. Na začátku vytváří strukturu \texttt{Server}, která připravena a naplněna daty pomocí funkce \texttt{server\_create()} a v nekonečném cyklu volá \texttt{server\_handle()}.

\subsubsection{\texttt{server.c}}
Tento soubor obsahuje všechny potřebné funkce pro spravování připojení serveru k jednotlivým klientům a správnou manipulaci s daty. Patří mezi ně:
\begin{itemize}
	\item \texttt{int server\_create(Server *s, char *config\_file)}

		Nejprve načte konfigurační soubor se jménem \texttt{config\_file}, který obsahuje IP adresu a port serveru a další hodnoty jako maximální počet připojených hráčů a vytvořených her v jeden moment. Dále vytvoří sever socket a připraví potřebné struktury pro server.
	\item \texttt{int server\_handle(Server *s)}

		Toto je základní funkce, která běží v \texttt{main.c} v nekonečné smyčce a obstarává komunikaci s klienty na nejnižší úrovni. Pomocí \texttt{select()} přijímá nové klienty a zprávy, které dále předává funkci \texttt{handle\_msg}. Dále také v intervalech posílá \texttt{PING} všem klientům a spravuje tak připojené a odpojené klienty.
	\item \texttt{int handle\_msg(Server *s, SEvent type, int fd, char *msg)}

		K zpracování zpráv a událostí slouží funkce \texttt{handle\_msg()}. Přijímá 3 různé typy událostí -- \texttt{CONNECT}, \texttt{MSG} a \texttt{DISCONNECT}. 

		Při připojení nového klienta je vytvořena nová struktura \texttt{Player}, která je přidána do pole \texttt{players[]} ve struktuře \texttt{Server}. Naopak při odpojení je z tohoto pole odstraněn, ale to jen v případě, že nebyl přihlášen pomocí \texttt{LOGIN}. Pokud se již hráč přihlásil, zůstává na serveru uložen pro budoucí přihlášení zpět pod stejným jménem.

		Pro zpracování jednotlivých zpráv jsou nadefinované pole, které obsahuje zprávu v podobě řetězce a ukazatel na funkci, která tuto zprávu obsluhuje. Díky tomu stačí zjistit, která z obslužná funkce patří k přijaté zprávě a následně ji zavolat. Zbytek přijaté zprávy je v každé obslužné funkci načítán pomocí \texttt{strtok()}, díky čemuž nemusí přebírat žádné parametry a všechny funkce mají stejný prototyp. Pokud je zpráva ve špatném formátu, zvedne se čítač nevalidních zpráv u daného hráče. Pokud překročí 10 nevalidních zpráv, je hráč odpojen.
	\item \texttt{int remove\_player(Server *s, Player *p)}

		Tato funkce slouží k odstranění struktury hráče ze serveru a uvolnění z paměti.
	\item \texttt{int remove\_game(Server *s, Game *g)}

		Tato funkce slouží k odstranění hry ze serveru a uvolnění z paměti.
\end{itemize}

\subsubsection{\texttt{handlers.c}}
Při zpracování zpráv všechny obslužné funkce kontrolují, jestli je hráč ve správném stavu stavového automatu z obrázku \ref{pic:automat}. Pokud je zpráva validní, je hráčův stav změněn na jiný dle předchozího stavu a typu události.

Pokud jsou předané argumenty zpráv chybné nebo se hráč nenachází ve správném stavu, je o tom informován chybovou zprávou \texttt{ERR}.

\begin{itemize}
	\item \texttt{int login\_handler(Server *s, Player *p)}

	Tato funkce slouží k přihlášení hráče na server. Načítá jeho jméno, u kterého ověřuje zda-li obsahuje pouze povolené znaky a jestli už na serveru neexistuje hráč se stejným jménem.
	\item \texttt{int create\_handler(Server *s, Player *p)}

	Funkce \texttt{int create\_handler()} slouží k vytvoření nové hry. Testuje, jestli název obsahuje pouze povolené znaky a jestli už na serveru neexistuje hra se stejným jménem. V případě úspěšného vytvoření hry je odesílatel zprávy přidán jako první hráč.

	\item \texttt{int join\_handler(Server *s, Player *p)}

	Pro zpracování zprávy k připojení do hry je \texttt{int join\_handler()}. Najde hru, jejíž název se shoduje s předaným názvem a zkontroluje zda-li hra není plná. Pokud vše sedí, hráč je připojen do hry.

	\item \texttt{int turn\_handler(Server *s, Player *p)}

	Tato funkce slouží ke zpracování zprávy o tahu hráče. Kontroluje, jestli je hráč na tahu, jestli je jeho tah v mezích herního pole a také že na zadané pozici už nebyl proveden jeden z předchozích tahů.
	\item \texttt{int leave\_handler(Server *s, Player *p)}
	\item \texttt{int load\_handler(Server *s, Player *p)}
	\item \texttt{int sync\_handler(Server *s, Player *p)}
	\item \texttt{int reconnect\_handler(Server *s, Player *p)}
\end{itemize}

\subsubsection{\texttt{game}}
\subsubsection{\texttt{utils}}

\subsection{Client}
Client je napsán v jazyce Python za použití základních knihoven, grafické knihovny \texttt{pygame} a GUI knihovny \texttt{pygame\_gui}. Program je rozdělen do scén, mezi kterými se přepíná. Při implemetaci byla snaha se držet MVC architektury, ale ta byla kvůli jednoduchosti většiny scén zachována jen v samotné scéně hry. 

Každá scéna dědí třídu \texttt{Scene}, která obsahuje základní metody na zpracování vstupu od uživatele a správného vykreslení všech komponent. Měnění mezi scénenami zajišťuje jednoduchá třída \texttt{SceneManager}.

Pro komunikaci se serverem je zde třída \texttt{Socket}, která za využitím knihovny \texttt{socket} zajišťuje komunikaci se serverem, frontu přijatých zpráv a metody zajišťující opětovné připojení při přerušení spojení. Tato třída běži v druhém vlákně pro zajištění plynulého chodu uživatelského rozhraní.

\end{document}

