\documentclass[11pt,a4paper]{article}

\usepackage{titling}
\usepackage{float}
\usepackage[parfill]{parskip}
\usepackage[utf8]{inputenc}
\usepackage[czech]{babel}
\usepackage[T1]{fontenc}

\usepackage{amsmath}
\usepackage{amsfonts}
\usepackage{amssymb}
\usepackage{graphicx}
\usepackage[left=2cm,right=2cm,top=2cm,bottom=2cm]{geometry}
% automata
\usepackage{tikz}
\usetikzlibrary {arrows.meta,automata,positioning}

\author{Vladan Trhlík}
\title{%
	Semestrální práce z KIV/UPS \\
	\large Dots and Boxes  \\
}

\begin{document}

\maketitle

\section{Popis hry}
Hra Dots and Boxes je strategická tahová hra pro dva hráče. Herní pole tvoří čtvercová mřížka teček (v tomto případě 4 × 4), a cílem hry je propojit sousední tečky tak, aby vznikl čtverec. Hráč, který uzavře čtverec, získá bod a pokračuje dalším tahem. Hra končí, jakmile jsou všechny čtverce uzavřeny, a vítězem se stává hráč, který získal více bodů.
\section{Protokol}

\subsection{Základní popis}
Při posílání zpráv mezi klienty a serverem se jednotlivé části zpráv oddělují znakem '\texttt{|}' a ukončovací znak je '\texttt{\textbackslash n}'. Jména hráčů a názvy her jsou omezena na velká a malá písmena a znak '\texttt{\_}'. Pokud není zpráva ve správném formátu, nebo zadané parametry nedávají v aktuálním kontextu smysl, odpovědí je zpráva \texttt{ERR<n>}, kde \texttt{n} je číslo jedné z chybových zpráv:
\begin{itemize}
	\item \texttt{1} -- invalid
	\item \texttt{2} -- name already exists
	\item \texttt{3} -- player not on turn
	\item \texttt{4} -- player not in game
	\item \texttt{5} -- already in game
	\item \texttt{6} -- max limit exceeded
\end{itemize}

\subsection{Zprávy}
\subsubsection*{Přihlášení}
\begin{itemize}
	\item \textbf{Client:} \texttt{LOGIN|<name>}
	\item \textbf{Server:} \texttt{OK / ERR<1/2>}
\end{itemize}

\subsubsection*{Načtení všech lobby}
\begin{itemize}
	\item \textbf{Client:} \texttt{LOAD}
	\item \textbf{Server:} \texttt{OK|<lobby1-name>|<lobby2-name>|...|<lobbyN-name>}
\end{itemize}

\subsubsection*{Vytvoření lobby}
\begin{itemize}
	\item \textbf{Client:} \texttt{CREATE|<lobby-name>}
	\item \textbf{Server:} \texttt{OK / ERR<1/2>}
\end{itemize}

\subsubsection*{Připojení do lobby}
\begin{itemize}
	\item \textbf{Client:} \texttt{JOIN|<lobby-name>}
	\item \textbf{Server:} \texttt{OK|<op-name> / ERR<1/5>}
\end{itemize}

\subsubsection*{Oponent se připojil}
\begin{itemize}
	\item \textbf{Server:} \texttt{OP\_JOIN|<op-name>}
	\item \textbf{Client:} \texttt{OK / ERR<4/5>}
\end{itemize}

\subsubsection*{Oponent se odpojil}
\begin{itemize}
	\item \textbf{Server:} \texttt{OP\_LEAVE}
	\item \textbf{Client:} \texttt{OK / ERR<4>}
\end{itemize}

\subsubsection*{Odpojení od hry}
\begin{itemize}
	\item \textbf{Client:} \texttt{LEAVE}
	\item \textbf{Server:} \texttt{OK / ERR<4>}
\end{itemize}

\subsubsection*{Tah}
\begin{itemize}
	\item \textbf{Client:} \texttt{TURN|X|Y}
	\item \textbf{Server:} \texttt{OK / ERR<1/3>}
\end{itemize}

\subsubsection*{Hráč je na tahu}
\begin{itemize}
	\item \textbf{Server:} \texttt{ON\_TURN}
	\item \textbf{Client:} \texttt{OK}
\end{itemize}

\subsubsection*{Oponent je na tahu}
\begin{itemize}
	\item \textbf{Server:} \texttt{OP\_TURN}
	\item \textbf{Client:} \texttt{OK}
\end{itemize}

\subsubsection*{Konec hry}
\begin{itemize}
	\item \textbf{Server:} \texttt{END|<WIN/LOSE>}
	\item \textbf{Client:} \texttt{OK}
\end{itemize}

\subsubsection*{Ping}
\begin{itemize}
	\item \textbf{Client:} \texttt{PING}
	\item \textbf{Server:} \texttt{PONG}
\end{itemize}

\subsubsection*{Synchronizace hry}
\begin{itemize}
	\item \textbf{Client:} \texttt{SYNC}
	\item \textbf{Server:} \texttt{OK|<w>|<h>|<stick-data>|<square-data> / ERR<4>}
	\\ data o hře budou reprezentována čísly 0-2, kde 0 je prázdné místo, 1 hráč, který žádá o SYNC a 2 jeho oponent
\end{itemize}

\section{Diagram}
\begin{tikzpicture}[node distance=4cm,on grid,auto] 

   \node[state,initial,initial text=$P_1$\ login] (p_0)   {lobby}; 
   \node[state,initial,initial text=$P_2$\ login] (p_1) [below=of p_0]  {lobby}; 
   \node[state] (q_0) [right=of p_0]  {waiting}; 
   \node[state] (q_1) [right=of q_0] {$P_1$ on turn}; 
   \node[state] (q_2) [right=of q_1] {$P_2$ on turn}; 
   \node[state] (q_3) [below=of q_1] {$P_1$ wins}; 
   \node[state] (q_4) [below=of q_2] {$P_2$ wins}; 
   \node[state] (q_5) [below=of q_3] {lobby}; 
    \path[->] 
		(p_0) edge node {create game \texttt{A}} (q_0)
		(p_1) edge [bend right] node {join game \texttt{A}} (q_1)
		(q_0) edge node {$P_2$ joined} (q_1)
		(q_1) edge [bend left] node {$P_1$ turn} (q_2)
			  edge [loop above] node {$P_1$ turn} ()
			  edge node {$P_1$ turn} (q_3)
		(q_2) edge [bend left] node {$P_2$ turn} (q_1)
			  edge [loop above] node {$P_2$ turn} ()
			  edge node {$P_2$ turn} (q_4)
		(q_3) edge node {leave} (q_5)
		(q_4) edge [bend left] node {leave} (q_5);
	\path[->,dashed]
		(q_5) edge [bend left] node {} (p_0)
		(q_5) edge [bend left] node {} (p_1)

\end{tikzpicture}
\end{document}
