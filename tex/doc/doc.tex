\documentclass[11pt,a4paper]{article}

\usepackage{titling}
\usepackage{float}
\usepackage[parfill]{parskip}
\usepackage[utf8]{inputenc}
\usepackage[czech]{babel}
\usepackage[T1]{fontenc}

\usepackage{amsmath}
\usepackage{amsfonts}
\usepackage{amssymb}
\usepackage{graphicx}
\usepackage[left=2cm,right=2cm,top=2cm,bottom=2cm]{geometry}
% automata
\usepackage{tikz}
\usetikzlibrary {arrows.meta,automata,positioning}

\author{Vladan Trhlík}
\title{%
	Dokumentace s semestrální práci z KIV/UPS \\
	\large Dots and Boxes  \\
}

\begin{document}

\maketitle

\section{Popis hry}
Hra Dots and Boxes je strategická tahová hra pro dva hráče. Herní pole tvoří čtvercová mřížka teček (v tomto případě 4 × 4), a cílem hry je propojit sousední tečky tak, aby vznikl čtverec. Hráč, který uzavře čtverec, získá bod a pokračuje dalším tahem. Hra končí, jakmile jsou všechny čtverce uzavřeny, a vítězem se stává hráč, který získal více bodů.
\section{Protokol}

\subsection{Základní popis}
Komunikace mezi klientem a serverem je zajištěna pomocí TCP protokolu. Při posílání zpráv mezi klienty a serverem se jednotlivé části zpráv oddělují znakem '\texttt{|}' a ukončovací znak je '\texttt{\textbackslash n}'. Jména hráčů a názvy her jsou omezena na velká a malá písmena a znak '\texttt{\_}'. Pokud není zpráva ve správném formátu, nebo zadané parametry nedávají v aktuálním kontextu smysl, odpovědí je zpráva \texttt{ERR<n>}, kde \texttt{n} je číslo jedné z chybových zpráv:
\begin{itemize}
	\item \texttt{1} -- invalid
	\item \texttt{2} -- name already exists
	\item \texttt{3} -- player not on turn
	\item \texttt{4} -- player not in game
	\item \texttt{5} -- already in game
	\item \texttt{6} -- max limit exceeded
\end{itemize}

\subsection{Zprávy}
\subsubsection*{Přihlášení}
\begin{itemize}
	\item \textbf{Klient:} \texttt{LOGIN|<name>}
	\item \textbf{Server:} \texttt{OK / ERR<1/2/6>}
	\item \texttt{<name>}: přihlašovací jméno hráče
\end{itemize}

\subsubsection*{Načtení všech her}
\begin{itemize}
	\item \textbf{Klient:} \texttt{LOAD}
	\item \textbf{Server:} \texttt{OK|<game1-name>|<game2-name>|...|<gameN-name>}
	\item \texttt{<game-name>}: název I-té hry
\end{itemize}

\subsubsection*{Vytvoření hry}
\begin{itemize}
	\item \textbf{Klient:} \texttt{CREATE|<game-name>}
	\item \textbf{Server:} \texttt{OK / ERR<1/2/6>}
	\item \texttt{<game-name>}: název hry
\end{itemize}

\subsubsection*{Připojení do hry}
\begin{itemize}
	\item \textbf{Klient:} \texttt{JOIN|<game-name>}
	\item \textbf{Server:} \texttt{OK|<op-name> / ERR<1/5/6>}
	\item \texttt{<game-name>}: název hry, \texttt{<op-name>}: jméno oponenta
\end{itemize}

\subsubsection*{Oponent se připojil}
\begin{itemize}
	\item \textbf{Server:} \texttt{OP\_JOIN|<op-name>}
	\item \textbf{Klient:} \texttt{OK / ERR<4/5>}
	\item \texttt{<op-name>}: jméno oponenta
\end{itemize}

\subsubsection*{Oponent se odpojil}
\begin{itemize}
	\item \textbf{Server:} \texttt{OP\_LEAVE}
	\item \textbf{Klient:} \texttt{OK / ERR<4>}
\end{itemize}

\subsubsection*{Odpojení od hry}
\begin{itemize}
	\item \textbf{Klient:} \texttt{LEAVE}
	\item \textbf{Server:} \texttt{OK / ERR<4>}
\end{itemize}

\subsubsection*{Tah}
\begin{itemize}
	\item \textbf{Server, Klient:} \texttt{TURN|<X>|<Y>}
	\item \textbf{Klient, Server:} \texttt{OK / ERR<1/3>}
	\item \texttt{<X>}, \texttt{<Y>}: pozice tahu
	\item Pokud přijde zpráva ze severu, jedná se o tah oponenta, pokud od klienta, posílají se data jeho oponentovi.
\end{itemize}

\subsubsection*{Obsazení čtverce}
\begin{itemize}
	\item \textbf{Server:} \texttt{(OP\_)ACQ|<X>|<Y>[|<X2>|<Y2>]}
	\item \textbf{Klient:} \texttt{OK}
	\item \texttt{<X>}, \texttt{<Y>}, \texttt{<X2>}, \texttt{<Y2>}: pozice čtverce
	\item \texttt{OP\_ACQ} informuje o obsazení oponentem, \texttt{ACQ} hráčem, který zprávu přijímá.
	\item Při propojení dvou teček může dojít k obsazení jednoho nebo dvou čtverců -- podle toho se pošle počet pozic čtverců.
\end{itemize}

\subsubsection*{Hráč je na tahu}
\begin{itemize}
	\item \textbf{Server:} \texttt{ON\_TURN}
	\item \textbf{Klient:} \texttt{OK}
\end{itemize}

\subsubsection*{Oponent je na tahu}
\begin{itemize}
	\item \textbf{Server:} \texttt{OP\_TURN}
	\item \textbf{Klient:} \texttt{OK}
\end{itemize}

\subsubsection*{Konec hry}
\begin{itemize}
	\item \textbf{Server:} \texttt{END|<WIN/LOSE>}
	\item \textbf{Klient:} \texttt{OK}
\end{itemize}

\subsubsection*{Ping}
\begin{itemize}
	\item \textbf{Klient:} \texttt{PING}, \texttt{PONG}
	\item \textbf{Server:} \texttt{PONG}, \texttt{PING}
\end{itemize}

\subsubsection*{Synchronizace stavu hry}
\begin{itemize}
	\item \textbf{Klient:} \texttt{SYNC}
	\item \textbf{Server:} \texttt{OK|<w>|<h>|<stick-data>|<square-data> / ERR<4>}
	\item \texttt{<w>}, \texttt{<h>}: velikost hracího pole, \texttt{<stick-data>}: data o propojených tečkách, \texttt{<square-data>}: data o obsazených čtvercích
	\item data o propojených tečkách a obsazených čtvercích jsou řetězce z čísel 0,1,2
	\begin{itemize}
		\item 0 = nepropojeno / neobsazeno
		\item 1 = hráč který žádá o \texttt{SYNC}
		\item 2 = oponent
	\end{itemize}
\end{itemize}

\subsubsection*{Připojení po výpadku spojení}
\begin{itemize}
	\item \textbf{Klient:} \texttt{RECONNECT|<name>|<game-name>}
	\item \textbf{Server:} \texttt{OK|<scene-code>}
	\item \texttt{<name>}: jméno hráče (nepovinné), \texttt{<game-name>}: název hry (nepovinné)
	\item \texttt{<scene-code>}: kód akce pro klienta (0 = login, 1 = lobby, 2 = game)
	\item Server se na základě údajů \texttt{<name>} a \texttt{<game-name>} rozhodne, zda-li se hráč připojí zpět do hry (2), nebo zůstane v lobby (1). Při neznámém jménu hráče je přesunut na nový login (0). Pokud byl hráč ve hře a připojí se zpět po více než 30 s, je automaticky přesunt do lobby.
\end{itemize}

\section{Diagram}
\begin{figure}[H]
\begin{tikzpicture}[node distance=4cm,on grid,auto] 

   \node[state,initial,initial text=$P_1$\ login] (p_0)   {lobby}; 
   \node[state,initial,initial text=$P_2$\ login] (p_1) [below=of p_0]  {lobby}; 
   \node[state] (q_0) [right=of p_0]  {waiting}; 
   \node[state] (q_1) [right=of q_0] {$P_1$ on turn}; 
   \node[state] (q_2) [right=of q_1] {$P_2$ on turn}; 
   \node[state] (q_3) [below=of q_1] {game end}; 
   \node[state] (q_5) [below=of q_3] {lobby}; 
    \path[->] 
		(p_0) edge node {create game \texttt{A}} (q_0)
		(p_1) edge [bend right] node {join game \texttt{A}} (q_1)
		(q_0) edge node {$P_2$ joined} (q_1)
		(q_1) edge [bend left] node {$P_1$ turn} (q_2)
			  edge [loop above] node {$P_1$ turn} ()
			  edge node {$P_1$ turn} (q_3)
		(q_2) edge [bend left] node {$P_2$ turn} (q_1)
			  edge [loop above] node {$P_2$ turn} ()
			  edge [bend left] node {$P_2$ turn} (q_3)
		(q_3) edge node {leave} (q_5);
	\path[->,dashed]
		(q_5) edge [bend left] node {} (p_0)
		(q_5) edge [bend left] node {} (p_1);

\end{tikzpicture}
\caption{Stavový automat představující cyklus hráče.}
\label{pic:automat}
\end{figure}

\section{Popis implementace}
Při implementaci bylo nutné sjednotit reprezentaci herního pole serveru a klienta. Pro uložení stavu hry bylo potřeba ukládat propojení teček a obsazení čtverců + kterému hráči dané pole patří. Reprezentace čtverců byla jednodušší, protože se dá jednoduše uložit na dvourozměrné pole o velikosti 3 $\times$ 3. 

K propojení teček by se dal použít například graf, ale ten by byl i při propojení všech sousedních teček velice řídký. Nakonec se i propojení teček reprezentuje dvourozměrných polem, ale počet prvků v jeho řádcích je proměnlivý. Pole má 7 řádků, kde sudé řádky mají 3 prvky (vodorovné spojnice) a liché 4 (svislé spojnice). Tato struktura je zachována i při synchronizaci herního pole pomocí \texttt{SYNC}, kde jsou přenášena data ve stejném pořadí jako jsou uložena v poli.

\subsection{Server}
Server je napsán jazyce C za použití standardních knihoven. Je rozdělen do několika modulů:
\begin{itemize}
	\item \texttt{main} -- vstupní bod programu
	\item \texttt{server} -- funkce pro spuštění serveru a obsluhy klientů
	\item \texttt{handlers} -- obslužné funkce pro jednotlivé typy zpráv
	\item \texttt{game} -- funkce pro vytváření hry, ověřování správnosti tahů apod.
	\item \texttt{utils} -- užitečné funkce pro odesílání zpráv a ověřování přechodů mezi stavy klientů
\end{itemize}
Všechny struktury jsou definovány ve \texttt{structs.h} -- obsahují strukturu \texttt{Game}, \texttt{Server} a \texttt{Player}, které slouží k uložení dat o jednlivých hráčích a hrách. Kromě toho se zde nachází i typy zpráv, výčty možných stavů a událostí, které jsou využity v přechodovém automatu (v \texttt{utils.c}).

Obsluha klientů běží jako jeden proces, který pomocí \texttt{select()} vybírá klienty, u kterých došlo ke změně nebo od nich přišla zpráva. Pokud od nějakého klienta nepřijde po nějakém časovém kvantu žádná zpráva, je jeho připojení ověřováno pomocí zprávy \texttt{PING}. Aby mohl jeden proces ověřovat připojení a obsluhovat klienty, je funkci \texttt{select()} nastaven timeout tak, aby se v rozumných intervalech kontrolovala doba od poslední zprávy jednotlivých klientů.

Pro každý typ zprávy je v modulu \texttt{handlers} obslužná funkce. Všechny obslužné funkce mají stejnou hlavičku, což umožňuje vytvořit pole těchto funkcí a pole řetězců definující jednotlivé typy zpráv. Díky tomu pak stačí ve smyčce projet všechny možné zprávy a při shodě zavolat příslušnou funkci.

Při zpracování zpráv všechny obslužné funkce kontrolují, jestli je hráč ve správném stavu stavového automatu z obrázku \ref{pic:automat}. Pokud je zpráva validní, je hráčův stav změněn na jiný dle předchozího stavu a typu události. Pokud jsou předané argumenty zpráv chybné nebo se hráč nenachází ve správném stavu, je o tom informován chybovou zprávou \texttt{ERR}.

\subsection{Klient}
Klient je napsán v jazyce Python za použití základních knihoven, grafické knihovny \texttt{pygame} a GUI knihovny \texttt{pygame\_gui}. Program je rozdělen do několika modulů:

\begin{itemize}
	\item \texttt{main} -- vstupní bod programu
	\item \texttt{user} -- třída pro ukládání dat o uživateli
	\item \texttt{scene}, \texttt{scene\_manager} -- základní třída scény + obsluha scén
	\item \texttt{mysocket} -- třída pro zajištění komunikace se serverem
	\item \texttt{login} -- scéna přihlášení
	\item \texttt{lobby} -- scéna lobby
	\item \texttt{game}, \texttt{game\_data}, \texttt{game\_view} -- scéna hry
\end{itemize}


Program je rozdělen do scén, mezi kterými se přepíná. Při implemetaci byla snaha se držet MVC architektury, ale ta byla kvůli jednoduchosti většiny scén zachována jen v samotné scéně hry. 

Každá scéna dědí třídu \texttt{Scene}, která obsahuje základní metody na zpracování vstupu od uživatele a správného vykreslení všech komponent. Měnění mezi scénenami zajišťuje jednoduchá třída \texttt{SceneManager}. Data o uživateli si jednotlivé scény předávají v instanci třídy \texttt{User}, která obsahuje všechna potřebné informace.

Pro komunikaci se serverem je zde třída \texttt{Socket}, která za využitím knihovny \texttt{socket} zajišťuje komunikaci se serverem, frontu přijatých zpráv a metody zajišťující opětovné připojení při přerušení spojení. Tato třída běži v druhém vlákně pro zajištění plynulého chodu uživatelského rozhraní.

\section{Sestavení a spuštění}
\subsection{Server}
Pro sestavení serveru je potřeba překladač \texttt{gcc} a nástroj k sestavení \texttt{make}. Samotný překlad je zahájen příkazen \texttt{make} v adresáři obsahující soubor \texttt{Makefile}. Po úspěšném dokončení by měl vzniknout spustitelný soubor \texttt{server}.

Pro nastavení serveru lze využít konfigurační soubor \texttt{config.txt}, který obsahuje 4 předdefinované proměnné a jejich hodnoty (odděleny mezerou, každá na jeden řádek) ilustrovány v tabulce \ref{tab:conf}.

\begin{table}
	\begin{center}
		\begin{tabular}[c]{l|l|l}
			proměnná & popis & příklad \\ \hline
			\texttt{maxPlayers} & maximální počet přihlášených hráčů & 10 \\
			\texttt{maxGames} & maximální počet her & 4 \\
			\texttt{port} & port, na kterém server naslouchá & 10000 \\
			\texttt{ip} & ip adresa, na kterém server naslouchá & 192.168.1.16 \\
		\end{tabular}
	\end{center}
	\caption{Proměnné konfiguračního souboru.}
	\label{tab:conf}
\end{table}

\subsection{Klient}
Pro spuštění klienta je potřeba mít nainstalovaný \texttt{Python} a knihovny \texttt{pygame} a \texttt{pygame\_gui}. Klient má při spuštění dva nepovinné parametry pro nastavení ip adresy a portu serveru, a spustí se takto:
$$ \texttt{python main.py [ip] [port]}$$
Pokud nejsou parametry zadány, jsou využity výchozí hodnoty (port = 10000, ip = localhost).

Po spuštění se hráč přihlásí pomocí unikátního uživatelského jména a kliknutím na tlačítko '\texttt{Login}'. Po úspěšném přihlášením se dostane do lobby, ve kterém může vytvořit novou hru nebo se může připojit do existující hry s jedním hráčem. Pro aktualizaci dostupných her je zde tlačítko '\texttt{U}'.

Po připojení do hry se hráči střídají v tazích. Na dolní liště jsou zobrazena jména hráčů a jejich skóre. Pokud je hráč na tahu, je jeho jméno podbarveno barevně. Když jsou všechny čtverce obsazené, na dolní liště se zobrazí výsledky (WIN/LOSE) a hráči se mohou vrátit zpět do lobby tlačítkem '\texttt{Leave}'.

Pokud se v průběhu hry hráč odpojí, může se připojit zpět do hry, ale jen v tom případě, že se přihlásí pod stejným uživatelským jménem. Při výpadku spojení se klient automaticky snaží o připojení zpět k serveru.

\section{Závěr}
S výsledkem semestrální práce jsem velmi spokojen. Tato práce byla mou první zkušeností s tvorbou programu využívajícího síťovou komunikaci, což mi umožnilo naučit se řadu nových věcí a získat cenné praktické zkušenosti.

Pro vývoj serveru jsem zvolil programovací jazyk \texttt{C}, protože mě tento jazyk baví a cítím se v něm velmi komfortně. S implementací serveru jsem spokojen, protože je navržen tak, aby byl snadno rozšiřitelný a~dokázal obsloužit až desítky klientů současně (větší zátěž nebyla testována). Jedním z možných zlepšení by mohl být podrobnější výpis stavu serveru, který by v přehledné tabulce zobrazoval informace o~připojených hráčích a probíhajících hrách.

Pro klienta jsem zvolil \texttt{Python}, díky jeho jednoduchosti, která usnadňuje rychlé prototypování, a také díky svým předchozím zkušenostem s knihovnou \texttt{pygame}. Aplikace klienta je navržena tak, aby poskytovala veškerou potřebnou funkcionalitu v co nejjednodušším a přehledném rozhraní. K dispozici je lišta, která vždy zobrazuje informace o připojení k serveru a jménu uživatele. Možným vylepšením by bylo zlepšení uživatelského rozhraní v lobby, kde by bylo vhodné lépe odlišit možnosti připojení k~existující hře od vytvoření nové. Dále by bylo dobré doplnit informaci o počtu hráčů v jednotlivých hrách, do kterých se může klient připojit.

\end{document}

